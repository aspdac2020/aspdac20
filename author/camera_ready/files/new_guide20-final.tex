\documentclass[conference]{IEEEtran}
\usepackage{geometry}
\geometry{letterpaper,left=1.75cm,right=1.75cm,top=1.5cm,bottom=1.5cm}
\begin{document}
%\IEEEoverridecommandlockouts

\title{\Large\bf Preparation GUIDE for Camera-Ready Manuscript\\~\\
	\large\bf Preparation of Papers in Two-Column Format\\
	for the ASP-DAC 2020 (\LaTeX version)}

	\author{\normalsize
	\begin{tabular}[t]{c@{\extracolsep{8em}}c}
		\large Author Name& \large Coauthor Name \\
		\\
		Author Department & Coauthor Department \\
		Author Institute  & Coauthor Institute \\
		City, ST~~zipcode & City, ST~~zipcode\\
		Tel: 123-456-7890 & Tel: +81-3-333-1234\\
		Fax: 123-456-0987 & Fax: +81-3-333-5678\\
		e-mail: aaa@bbb.ccc.ddd & e-mail eee@ffff.ggg.hh\\
\end{tabular}}

% use for special paper notices
%\IEEEspecialpapernotice{(Invited Paper)}

% make the title area
\maketitle



\makeatletter
\def\ps@IEEEtitlepagestyle{%
  \def\@oddfoot{\mycopyrightnotice}%
  \def\@evenfoot{}%
}
\makeatother
\def\mycopyrightnotice{%
  \begin{minipage}{\textwidth}
    \footnotesize
    xxx-x-xxxx-xxxx-x/xx/\$31.00~\copyright~20xx IEEE \hfill\\~\\
  \end{minipage}
  \gdef\mycopyrightnotice{}% just in case
}





{\small\bf Abstract---
Abstract is a brief (50-80 word) synopsis of your paper.
The purpose is to provide a quick outline of your
presentation, giving the reader an overview of the
research. It must be fit within the size allowed, which is
about 3 inches or 7.5 centimeters.}




\section{Introduction}

These introductions give you basic guidelines for preparing 
camera-ready papers for the ASP-DAC 2020.

The instructions assume that you have computer that can generate an 
IEEE Xplore-compatible PDF file or any source file(s) which can be 
converted to PDF using IEEE PDF eXpress. 

These instructions have been prepared in the preferred format. For items
not addressed here, please refer to recent issues of IEEE Transactions
and simulate, as closely as possible.

\section{How to Format the Page}

\subsection{Paper Format} 
Prepare camera-ready paper in full size format, on 8.5'' $\times$ 11'' 
(US Letter size) paper.

{\bf The width and height of the main text (including figures, tables, and 
	footnotes, if any) is 181 mm (7.13 inches) and 250 mm (9.84 inches), 
	respectively.}  Adjust the margins around the main text as follows;

\begin{itemize}
	\item Top and bottom margins: 14.7 mm (0.58 inches).
	\item Left and right margins: 17.45 mm (0.685 inches).
\end{itemize}

There are two columns, 88 mm (3.465 inches) wide each, with 5 mm (0.2 inches) 
space between them (i.e., $88\mbox{ mm}\times2+5\mbox{ mm}=181\mbox{ mm}$).

You should left- and right-justify your columns. On the last page of
your paper, try to adjust the lengths of the two columns so that they
are the same. Use automatic hyphenation if you have it and check
spelling.

%{\bf Number each of you submitted pages at the top, right corner, in
%non-photographic light blue pencil.}

\subsection{Fonts}
The best results will be obtained if your computer word-processor has
several font sizes. Try to follow the font sizes specified in Table I as
best as you can. As an aid to gauging font size, 1 point is about
0.35 mm. Use a proportional, serif font such as Times of Dutch Roman.

\begin{table}[tb]
	\caption{Fonts for Camera-Ready Papers}
	\begin{minipage}{8cm}
		\def\arraystretch{1.5}\tabcolsep 2pt
		\def\thefootnote{a}\footnotesize
		\begin{tabular}{l@{~}l@{~~~}l}
			\hline
			\parbox[c]{7mm}{Font\newline Size} & Style & Text\\
			\hline
			14pt&bold     &Paper title\\
			12pt&         &Authors' names\\
			10pt&         &Authors' affiliations, main text, equations,\\[-5pt]
			&         &first letters in section titles\footnotemark[1]\\
			10pt&italic   &Subheddings\\
			~9pt&bold     &Abstract\\
			~8pt&         &Section titles\footnotemark[1], table
			names\footnotemark[1], first letters in table\\[-5pt]
			&         &captions\footnotemark[1],
			tables, figure captions, references,\\[-5pt]
			&         &footnotes, text subscripts and superscripts\\
			~6pt&         &Table captions\footnotemark[1], table superscripts\\
			\hline
		\end{tabular}
		\footnotetext[1]{Uppercase}
	\end{minipage}
\end{table}

\section{Figures and Tables}
Position figures and tables at the tops and bottoms of columns. Avoid
placing them in the middle of columns. Large figures and tables may
span across both columns. Figure captions should be below the figures;
table captions should be above the tables. Avoid placing figures and
tables before their first mention in the text. Use the abbreviation
``Fig.1'', even at the beginning of a sentence.

\begin{figure}[tb]
	\begin{center}
		\begin{minipage}{5cm}
			$.$\hrulefill $.$\\$|$\hfill $|$\\$|$\hfill $|$\\$|$\hfill $|$\\
			$|$\hfill this is \hfill $|$\\
			$|$\hfill a sample \hfill $|$\\
			$|$\hfill  figure  \hfill $|$\\
			$|$\hfill $|$\\$|$\hfill $|$\\$|$\hfill $|$\\$.$\hrulefill $.$\\
		\end{minipage}
		\caption{This is a sample figure. Captions exceeding
			one line are arranged like this.}
	\end{center}
\end{figure}

{\bf
	To meet the requirements for IEEE Xplore, the paper's graphics
	should have resolutions of 600dpi for monochrome, 300 dpi for 
	grayscale, and 300 dpi for color.}

\section{Helpful Hints}

\subsection{References}
List and number all references at the end of the paper. When referring
to them in the text, type the corresponding reference number in the
parentheses as shown at the end of this sentence \cite{key}. Number
the citations consecutively. The sentence punctuation follows the
parentheses. Do not use ``Ref.\cite{baz}'' or
``reference\cite{baz}'' except at the beginning of a sentence.

\subsection{Footnotes}
Number the footnotes separately in superscripts. Place the actual
footnote at the bottom of the column in which it is cited. Do not put
footnotes in the reference list.

\subsection{Authors names}

Give all authors' names; do not use ``et al'' unless there are six
authors or more. Papers that have not been published, even if they have
been submitted for publication, should be cited as
``unpublished''\cite{unpub}.  Papers that have been accepted for
publication should be cited as ``in press''\cite{inpress}.
Capitalize only the first word in a paper title, except for proper nouns
and element symbols.

For papers published in translation journals, please give the English
citation first, followed by the original foreign language
citations\cite{trans}.

\subsection{Notice for \LaTeX\ users}

If you use \LaTeX\ to create your camera-ready paper, we recommend you
to use IEEE PDF eXpress (or {\tt dvipdfm}) to produce PDF files from dvi 
files. If you cannot use it, please use Type1 fonts instead of ugly Type3 fonts!

\section{Summary and Conclusions}

This template can be downloaded through the ASP-DAC 2020 web site
(http://www.aspdac.com/). If you have any problem, please contact ASP-DAC
2020 publication chair\\
(wei.zhang@ust.hk).

%this is how to do an unnumbered subsection 
\section*{\sc Acknowledgments}
This article was written by referring to {\em ``Author's guide --
	Preparation of Papers in Two-Column Format for VLSI Symposia on
	Technology and Circuits''}, the {\em ``Preparation of Papers in
	Two-Column Format for the Proceedings of the 32nd ACM/IEEE Design
	Automation Conference''} written by Ann Burgmeyer, IEEE and {\em ``the
	template for producing IEEE-format articles using \LaTeX''}, written by
Matthew Ward, Worcester Polytechnic Institute.

\begin{thebibliography}{9}
	\footnotesize
	\bibitem{key}
	I. M. Author,
	``Some related article I wrote,''
	{\em Some Fine Journal}, vol. 17, pp. 1--100, 1987.
	
	\bibitem{baz}
	A. N. Expert,
	{\em A Book He Wrote,}
	His Publisher, 1989.
	
	\bibitem{unpub}
	M. Smith,
	``Title of paper optional here,''
	unpublished.
	
	\bibitem{inpress}
	K. Rose,
	``Title of paper with only first word capitalized,''	% bug fixed by M. Imai
	in press.
	
	\bibitem{trans}
	T. Murayama,
	``Title of paper published in translation journals,''	% bug fixed by M. Imai
	{\em Some English Journal}, vol. 17, pp. 1--100, 1995.	% bug fixed by M. Imai
	({\em Original Foreign Journal, vol. 1, pp. 100-200, 1993}.)	% ditto
	
\end{thebibliography}

\end{document}
